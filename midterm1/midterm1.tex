% Options for packages loaded elsewhere
\PassOptionsToPackage{unicode}{hyperref}
\PassOptionsToPackage{hyphens}{url}
%
\documentclass[
]{article}
\usepackage{amsmath,amssymb}
\usepackage{iftex}
\ifPDFTeX
  \usepackage[T1]{fontenc}
  \usepackage[utf8]{inputenc}
  \usepackage{textcomp} % provide euro and other symbols
\else % if luatex or xetex
  \usepackage{unicode-math} % this also loads fontspec
  \defaultfontfeatures{Scale=MatchLowercase}
  \defaultfontfeatures[\rmfamily]{Ligatures=TeX,Scale=1}
\fi
\usepackage{lmodern}
\ifPDFTeX\else
  % xetex/luatex font selection
\fi
% Use upquote if available, for straight quotes in verbatim environments
\IfFileExists{upquote.sty}{\usepackage{upquote}}{}
\IfFileExists{microtype.sty}{% use microtype if available
  \usepackage[]{microtype}
  \UseMicrotypeSet[protrusion]{basicmath} % disable protrusion for tt fonts
}{}
\makeatletter
\@ifundefined{KOMAClassName}{% if non-KOMA class
  \IfFileExists{parskip.sty}{%
    \usepackage{parskip}
  }{% else
    \setlength{\parindent}{0pt}
    \setlength{\parskip}{6pt plus 2pt minus 1pt}}
}{% if KOMA class
  \KOMAoptions{parskip=half}}
\makeatother
\usepackage{xcolor}
\usepackage[margin=1in]{geometry}
\usepackage{color}
\usepackage{fancyvrb}
\newcommand{\VerbBar}{|}
\newcommand{\VERB}{\Verb[commandchars=\\\{\}]}
\DefineVerbatimEnvironment{Highlighting}{Verbatim}{commandchars=\\\{\}}
% Add ',fontsize=\small' for more characters per line
\usepackage{framed}
\definecolor{shadecolor}{RGB}{248,248,248}
\newenvironment{Shaded}{\begin{snugshade}}{\end{snugshade}}
\newcommand{\AlertTok}[1]{\textcolor[rgb]{0.94,0.16,0.16}{#1}}
\newcommand{\AnnotationTok}[1]{\textcolor[rgb]{0.56,0.35,0.01}{\textbf{\textit{#1}}}}
\newcommand{\AttributeTok}[1]{\textcolor[rgb]{0.13,0.29,0.53}{#1}}
\newcommand{\BaseNTok}[1]{\textcolor[rgb]{0.00,0.00,0.81}{#1}}
\newcommand{\BuiltInTok}[1]{#1}
\newcommand{\CharTok}[1]{\textcolor[rgb]{0.31,0.60,0.02}{#1}}
\newcommand{\CommentTok}[1]{\textcolor[rgb]{0.56,0.35,0.01}{\textit{#1}}}
\newcommand{\CommentVarTok}[1]{\textcolor[rgb]{0.56,0.35,0.01}{\textbf{\textit{#1}}}}
\newcommand{\ConstantTok}[1]{\textcolor[rgb]{0.56,0.35,0.01}{#1}}
\newcommand{\ControlFlowTok}[1]{\textcolor[rgb]{0.13,0.29,0.53}{\textbf{#1}}}
\newcommand{\DataTypeTok}[1]{\textcolor[rgb]{0.13,0.29,0.53}{#1}}
\newcommand{\DecValTok}[1]{\textcolor[rgb]{0.00,0.00,0.81}{#1}}
\newcommand{\DocumentationTok}[1]{\textcolor[rgb]{0.56,0.35,0.01}{\textbf{\textit{#1}}}}
\newcommand{\ErrorTok}[1]{\textcolor[rgb]{0.64,0.00,0.00}{\textbf{#1}}}
\newcommand{\ExtensionTok}[1]{#1}
\newcommand{\FloatTok}[1]{\textcolor[rgb]{0.00,0.00,0.81}{#1}}
\newcommand{\FunctionTok}[1]{\textcolor[rgb]{0.13,0.29,0.53}{\textbf{#1}}}
\newcommand{\ImportTok}[1]{#1}
\newcommand{\InformationTok}[1]{\textcolor[rgb]{0.56,0.35,0.01}{\textbf{\textit{#1}}}}
\newcommand{\KeywordTok}[1]{\textcolor[rgb]{0.13,0.29,0.53}{\textbf{#1}}}
\newcommand{\NormalTok}[1]{#1}
\newcommand{\OperatorTok}[1]{\textcolor[rgb]{0.81,0.36,0.00}{\textbf{#1}}}
\newcommand{\OtherTok}[1]{\textcolor[rgb]{0.56,0.35,0.01}{#1}}
\newcommand{\PreprocessorTok}[1]{\textcolor[rgb]{0.56,0.35,0.01}{\textit{#1}}}
\newcommand{\RegionMarkerTok}[1]{#1}
\newcommand{\SpecialCharTok}[1]{\textcolor[rgb]{0.81,0.36,0.00}{\textbf{#1}}}
\newcommand{\SpecialStringTok}[1]{\textcolor[rgb]{0.31,0.60,0.02}{#1}}
\newcommand{\StringTok}[1]{\textcolor[rgb]{0.31,0.60,0.02}{#1}}
\newcommand{\VariableTok}[1]{\textcolor[rgb]{0.00,0.00,0.00}{#1}}
\newcommand{\VerbatimStringTok}[1]{\textcolor[rgb]{0.31,0.60,0.02}{#1}}
\newcommand{\WarningTok}[1]{\textcolor[rgb]{0.56,0.35,0.01}{\textbf{\textit{#1}}}}
\usepackage{graphicx}
\makeatletter
\def\maxwidth{\ifdim\Gin@nat@width>\linewidth\linewidth\else\Gin@nat@width\fi}
\def\maxheight{\ifdim\Gin@nat@height>\textheight\textheight\else\Gin@nat@height\fi}
\makeatother
% Scale images if necessary, so that they will not overflow the page
% margins by default, and it is still possible to overwrite the defaults
% using explicit options in \includegraphics[width, height, ...]{}
\setkeys{Gin}{width=\maxwidth,height=\maxheight,keepaspectratio}
% Set default figure placement to htbp
\makeatletter
\def\fps@figure{htbp}
\makeatother
\setlength{\emergencystretch}{3em} % prevent overfull lines
\providecommand{\tightlist}{%
  \setlength{\itemsep}{0pt}\setlength{\parskip}{0pt}}
\setcounter{secnumdepth}{-\maxdimen} % remove section numbering
\ifLuaTeX
  \usepackage{selnolig}  % disable illegal ligatures
\fi
\IfFileExists{bookmark.sty}{\usepackage{bookmark}}{\usepackage{hyperref}}
\IfFileExists{xurl.sty}{\usepackage{xurl}}{} % add URL line breaks if available
\urlstyle{same}
\hypersetup{
  pdftitle={Midterm 1 W24},
  pdfauthor={Spencer Bryan},
  hidelinks,
  pdfcreator={LaTeX via pandoc}}

\title{Midterm 1 W24}
\author{Spencer Bryan}
\date{2024-02-06}

\begin{document}
\maketitle

\hypertarget{instructions}{%
\subsection{Instructions}\label{instructions}}

Answer the following questions and complete the exercises in RMarkdown.
Please embed all of your code and push your final work to your
repository. Your code must be organized, clean, and run free from
errors. Remember, you must remove the \texttt{\#} for any included code
chunks to run. Be sure to add your name to the author header above.

Your code must knit in order to be considered. If you are stuck and
cannot answer a question, then comment out your code and knit the
document. You may use your notes, labs, and homework to help you
complete this exam. Do not use any other resources- including AI
assistance.

Don't forget to answer any questions that are asked in the prompt!

Be sure to push your completed midterm to your repository. This exam is
worth 30 points.

\hypertarget{background}{%
\subsection{Background}\label{background}}

In the data folder, you will find data related to a study on wolf
mortality collected by the National Park Service. You should start by
reading the \texttt{README\_NPSwolfdata.pdf} file. This will provide an
abstract of the study and an explanation of variables.

The data are from: Cassidy, Kira et al.~(2022). Gray wolf packs and
human-caused wolf mortality.
\href{https://doi.org/10.5061/dryad.mkkwh713f}{Dryad}.

\hypertarget{load-the-libraries}{%
\subsection{Load the libraries}\label{load-the-libraries}}

\begin{Shaded}
\begin{Highlighting}[]
\FunctionTok{library}\NormalTok{(}\StringTok{"tidyverse"}\NormalTok{)}
\FunctionTok{library}\NormalTok{(}\StringTok{"janitor"}\NormalTok{)}
\end{Highlighting}
\end{Shaded}

\hypertarget{load-the-wolves-data}{%
\subsection{Load the wolves data}\label{load-the-wolves-data}}

In these data, the authors used \texttt{NULL} to represent missing
values. I am correcting this for you below and using \texttt{janitor} to
clean the column names.

\begin{Shaded}
\begin{Highlighting}[]
\NormalTok{wolves }\OtherTok{\textless{}{-}} \FunctionTok{read.csv}\NormalTok{(}\StringTok{"data/NPS\_wolfmortalitydata.csv"}\NormalTok{, }\AttributeTok{na =} \FunctionTok{c}\NormalTok{(}\StringTok{"NULL"}\NormalTok{)) }\SpecialCharTok{\%\textgreater{}\%} \FunctionTok{clean\_names}\NormalTok{()}
\end{Highlighting}
\end{Shaded}

\hypertarget{questions}{%
\subsection{Questions}\label{questions}}

Problem 1. (1 point) Let's start with some data exploration. What are
the variable (column) names?

\begin{Shaded}
\begin{Highlighting}[]
\FunctionTok{names}\NormalTok{(wolves)}
\end{Highlighting}
\end{Shaded}

\begin{verbatim}
##  [1] "park"         "biolyr"       "pack"         "packcode"     "packsize_aug"
##  [6] "mort_yn"      "mort_all"     "mort_lead"    "mort_nonlead" "reprody1"    
## [11] "persisty1"
\end{verbatim}

Problem 2. (1 point) Use the function of your choice to summarize the
data and get an idea of its structure.

\begin{Shaded}
\begin{Highlighting}[]
\FunctionTok{glimpse}\NormalTok{(wolves)}
\end{Highlighting}
\end{Shaded}

\begin{verbatim}
## Rows: 864
## Columns: 11
## $ park         <chr> "DENA", "DENA", "DENA", "DENA", "DENA", "DENA", "DENA", "~
## $ biolyr       <int> 1996, 1991, 2017, 1996, 1992, 1994, 2007, 2007, 1995, 200~
## $ pack         <chr> "McKinley River1", "Birch Creek N", "Eagle Gorge", "East ~
## $ packcode     <int> 89, 58, 71, 72, 74, 77, 101, 108, 109, 53, 63, 66, 70, 72~
## $ packsize_aug <dbl> 12, 5, 8, 13, 7, 6, 10, NA, 9, 8, 7, 11, 0, 19, 15, 12, 1~
## $ mort_yn      <int> 1, 1, 1, 1, 1, 1, 1, 1, 1, 1, 1, 1, 1, 1, 1, 1, 1, 1, 1, ~
## $ mort_all     <int> 4, 2, 2, 2, 2, 2, 2, 2, 2, 1, 1, 1, 1, 1, 1, 1, 1, 1, 1, ~
## $ mort_lead    <int> 2, 2, 0, 0, 0, 0, 1, 2, 1, 1, 1, 0, 0, 0, 1, 1, 1, 0, 0, ~
## $ mort_nonlead <int> 2, 0, 2, 2, 2, 2, 1, 0, 1, 0, 0, 1, 1, 1, 0, 0, 0, 1, 1, ~
## $ reprody1     <int> 0, 0, NA, 1, NA, 0, 0, 1, 0, 1, 0, 1, 0, 1, 1, 1, 1, 1, 1~
## $ persisty1    <int> 0, 0, 1, 1, 1, 1, 0, 1, 0, 1, 0, 1, 0, 1, 1, 1, 1, 1, 1, ~
\end{verbatim}

Problem 3. (3 points) Which parks/ reserves are represented in the data?
Don't just use the abstract, pull this information from the data.

\begin{Shaded}
\begin{Highlighting}[]
\NormalTok{wolves }\SpecialCharTok{\%\textgreater{}\%}
  \FunctionTok{tabyl}\NormalTok{(park)}
\end{Highlighting}
\end{Shaded}

\begin{verbatim}
##  park   n    percent
##  DENA 340 0.39351852
##  GNTP  77 0.08912037
##   VNP  48 0.05555556
##   YNP 248 0.28703704
##  YUCH 151 0.17476852
\end{verbatim}

\begin{verbatim}
  Denali National Park and Preserve (DENA), Grand Teton National Park (GTNP), Voyageurs National Park (VNP), Yellowstone National Park(YNP), and Yukon-Charley Rivers National Preserve (YUCH) are represented in this data.
\end{verbatim}

Problem 4. (4 points) Which park has the largest number of wolf packs?

\begin{Shaded}
\begin{Highlighting}[]
\NormalTok{wolves }\SpecialCharTok{\%\textgreater{}\%}
  \FunctionTok{group\_by}\NormalTok{(park) }\SpecialCharTok{\%\textgreater{}\%}
  \FunctionTok{select}\NormalTok{(park, packcode) }\SpecialCharTok{\%\textgreater{}\%}
  \FunctionTok{summarise}\NormalTok{(}\AttributeTok{number\_packs=}\FunctionTok{n\_distinct}\NormalTok{(packcode))}\SpecialCharTok{\%\textgreater{}\%}
  \FunctionTok{arrange}\NormalTok{(}\FunctionTok{desc}\NormalTok{(number\_packs))}
\end{Highlighting}
\end{Shaded}

\begin{verbatim}
## # A tibble: 5 x 2
##   park  number_packs
##   <chr>        <int>
## 1 DENA            69
## 2 YNP             46
## 3 YUCH            36
## 4 VNP             22
## 5 GNTP            12
\end{verbatim}

\begin{verbatim}
  Denali Nat'l Park and preserve has the largest number of unique wolfpacks 
\end{verbatim}

Problem 5. (4 points) Which park has the highest total number of
human-caused mortalities \texttt{mort\_all}?

\begin{Shaded}
\begin{Highlighting}[]
\NormalTok{wolves }\SpecialCharTok{\%\textgreater{}\%}
  \FunctionTok{group\_by}\NormalTok{(park) }\SpecialCharTok{\%\textgreater{}\%}
  \FunctionTok{select}\NormalTok{(park, mort\_all) }\SpecialCharTok{\%\textgreater{}\%}
  \FunctionTok{summarise}\NormalTok{(}\AttributeTok{total\_mort=} \FunctionTok{sum}\NormalTok{(mort\_all)) }\SpecialCharTok{\%\textgreater{}\%}
  \FunctionTok{arrange}\NormalTok{(}\FunctionTok{desc}\NormalTok{(total\_mort))}
\end{Highlighting}
\end{Shaded}

\begin{verbatim}
## # A tibble: 5 x 2
##   park  total_mort
##   <chr>      <int>
## 1 YUCH         136
## 2 YNP           72
## 3 DENA          64
## 4 GNTP          38
## 5 VNP           11
\end{verbatim}

\begin{verbatim}
Yukon-Rivers Charlie Nat'l Preserve (YUCH) has the highest total human-caused mortalities.
\end{verbatim}

The wolves in
\href{https://www.nps.gov/yell/learn/nature/wolf-restoration.htm}{Yellowstone
National Park} are an incredible conservation success story. Let's focus
our attention on this park.

Problem 6. (2 points) Create a new object ``ynp'' that only includes the
data from Yellowstone National Park.

\begin{Shaded}
\begin{Highlighting}[]
\NormalTok{ynp }\OtherTok{\textless{}{-}}\NormalTok{ wolves }\SpecialCharTok{\%\textgreater{}\%}
  \FunctionTok{filter}\NormalTok{(park }\SpecialCharTok{==} \StringTok{"YNP"}\NormalTok{)}
\end{Highlighting}
\end{Shaded}

Problem 7. (3 points) Among the Yellowstone wolf packs, the
\href{https://www.pbs.org/wnet/nature/in-the-valley-of-the-wolves-the-druid-wolf-pack-story/209/}{Druid
Peak Pack} is one of most famous. What was the average pack size of this
pack for the years represented in the data?

\begin{Shaded}
\begin{Highlighting}[]
\NormalTok{ynp }\SpecialCharTok{\%\textgreater{}\%}
  \FunctionTok{select}\NormalTok{(pack, packsize\_aug) }\SpecialCharTok{\%\textgreater{}\%}
  \FunctionTok{filter}\NormalTok{(pack }\SpecialCharTok{==} \StringTok{"druid"}\NormalTok{) }\SpecialCharTok{\%\textgreater{}\%}
  \FunctionTok{summarise}\NormalTok{(}\AttributeTok{avg\_packsize=} \FunctionTok{mean}\NormalTok{(packsize\_aug))}
\end{Highlighting}
\end{Shaded}

\begin{verbatim}
##   avg_packsize
## 1     13.93333
\end{verbatim}

Problem 8. (4 points) Pack dynamics can be hard to predict- even for
strong packs like the Druid Peak pack. At which year did the Druid Peak
pack have the largest pack size? What do you think happened in 2010?

This filters to find the largest August pack size recorded for the
``Druid'' pack

\begin{Shaded}
\begin{Highlighting}[]
\NormalTok{ynp }\SpecialCharTok{\%\textgreater{}\%}
  \FunctionTok{select}\NormalTok{(pack, packsize\_aug, biolyr) }\SpecialCharTok{\%\textgreater{}\%}
  \FunctionTok{filter}\NormalTok{(pack }\SpecialCharTok{==} \StringTok{"druid"}\NormalTok{) }\SpecialCharTok{\%\textgreater{}\%}
  \FunctionTok{summarise}\NormalTok{(}\AttributeTok{largest\_packsize =} \FunctionTok{max}\NormalTok{(packsize\_aug))}
\end{Highlighting}
\end{Shaded}

\begin{verbatim}
##   largest_packsize
## 1               37
\end{verbatim}

\begin{verbatim}
This displays the pertinent related data for the maximum August pack size found.
\end{verbatim}

\begin{Shaded}
\begin{Highlighting}[]
\NormalTok{ynp }\SpecialCharTok{\%\textgreater{}\%}
  \FunctionTok{select}\NormalTok{(pack, biolyr, packsize\_aug) }\SpecialCharTok{\%\textgreater{}\%}
  \FunctionTok{filter}\NormalTok{(pack }\SpecialCharTok{==} \StringTok{"druid"}\NormalTok{) }\SpecialCharTok{\%\textgreater{}\%}
  \FunctionTok{filter}\NormalTok{(packsize\_aug }\SpecialCharTok{==} \DecValTok{37}\NormalTok{)}
\end{Highlighting}
\end{Shaded}

\begin{verbatim}
##    pack biolyr packsize_aug
## 1 druid   2001           37
\end{verbatim}

\begin{verbatim}
View data for 2010 for the "druid" pack
\end{verbatim}

\begin{Shaded}
\begin{Highlighting}[]
\NormalTok{ynp }\SpecialCharTok{\%\textgreater{}\%}
  \FunctionTok{select}\NormalTok{(pack, biolyr, packsize\_aug) }\SpecialCharTok{\%\textgreater{}\%}
  \FunctionTok{filter}\NormalTok{(pack }\SpecialCharTok{==} \StringTok{"druid"}\NormalTok{) }\SpecialCharTok{\%\textgreater{}\%}
  \FunctionTok{filter}\NormalTok{(biolyr }\SpecialCharTok{==} \DecValTok{2010}\NormalTok{)}
\end{Highlighting}
\end{Shaded}

\begin{verbatim}
##    pack biolyr packsize_aug
## 1 druid   2010            0
\end{verbatim}

\begin{verbatim}
It is likely that the pack died off during at some point between the 2010 and 2009 August pack size counting (the 2009 biological year). There is no data observed after 2010, so a likely conclusion is that the "druid" pack died off. The data shown by the code chunk below shows the data to support this.
\end{verbatim}

\begin{Shaded}
\begin{Highlighting}[]
\NormalTok{ynp }\SpecialCharTok{\%\textgreater{}\%}
  \FunctionTok{select}\NormalTok{(pack, biolyr, packsize\_aug) }\SpecialCharTok{\%\textgreater{}\%}
  \FunctionTok{filter}\NormalTok{(pack }\SpecialCharTok{==} \StringTok{"druid"}\NormalTok{) }\SpecialCharTok{\%\textgreater{}\%}
  \FunctionTok{arrange}\NormalTok{(}\FunctionTok{desc}\NormalTok{(biolyr))}
\end{Highlighting}
\end{Shaded}

\begin{verbatim}
##     pack biolyr packsize_aug
## 1  druid   2010            0
## 2  druid   2009           12
## 3  druid   2008           21
## 4  druid   2007           18
## 5  druid   2006           15
## 6  druid   2005            5
## 7  druid   2004           13
## 8  druid   2003           18
## 9  druid   2002           16
## 10 druid   2001           37
## 11 druid   2000           27
## 12 druid   1999            9
## 13 druid   1998            8
## 14 druid   1997            5
## 15 druid   1996            5
\end{verbatim}

Problem 9. (5 points) Among the YNP wolf packs, which one has had the
highest overall persistence \texttt{persisty1} for the years represented
in the data? Look this pack up online and tell me what is unique about
its behavior- specifically, what prey animals does this pack specialize
on?

\begin{Shaded}
\begin{Highlighting}[]
\CommentTok{\# coded based on total persistence over time by summing the amount of times the pack has remained in it\textquotesingle{}s same spot.}
\NormalTok{ynp }\SpecialCharTok{\%\textgreater{}\%}
  \FunctionTok{select}\NormalTok{(pack, packcode, persisty1) }\SpecialCharTok{\%\textgreater{}\%}
  \FunctionTok{group\_by}\NormalTok{(packcode) }\SpecialCharTok{\%\textgreater{}\%}
  \FunctionTok{summarise}\NormalTok{(}\AttributeTok{total\_pack\_persisty=} \FunctionTok{sum}\NormalTok{(persisty1)) }\SpecialCharTok{\%\textgreater{}\%}
  \FunctionTok{arrange}\NormalTok{(}\FunctionTok{desc}\NormalTok{(total\_pack\_persisty))}
\end{Highlighting}
\end{Shaded}

\begin{verbatim}
## # A tibble: 46 x 2
##    packcode total_pack_persisty
##       <int>               <int>
##  1       38                  26
##  2       24                  20
##  3       52                  18
##  4       36                  12
##  5       15                  10
##  6       11                   9
##  7       20                   9
##  8       29                   9
##  9       33                   8
## 10       34                   8
## # i 36 more rows
\end{verbatim}

\begin{Shaded}
\begin{Highlighting}[]
\CommentTok{\# to determine the name of pack 38}
\NormalTok{ynp }\SpecialCharTok{\%\textgreater{}\%}
  \FunctionTok{select}\NormalTok{(pack, packcode) }\SpecialCharTok{\%\textgreater{}\%}
  \FunctionTok{filter}\NormalTok{(packcode }\SpecialCharTok{==} \DecValTok{38}\NormalTok{)}
\end{Highlighting}
\end{Shaded}

\begin{verbatim}
##       pack packcode
## 1  mollies       38
## 2  mollies       38
## 3  mollies       38
## 4  mollies       38
## 5  mollies       38
## 6  mollies       38
## 7  mollies       38
## 8  mollies       38
## 9  mollies       38
## 10 mollies       38
## 11 mollies       38
## 12 mollies       38
## 13 mollies       38
## 14 mollies       38
## 15 mollies       38
## 16 mollies       38
## 17 mollies       38
## 18 mollies       38
## 19 mollies       38
## 20 mollies       38
## 21 mollies       38
## 22 mollies       38
## 23 mollies       38
## 24 mollies       38
## 25 mollies       38
## 26 mollies       38
\end{verbatim}

\begin{verbatim}
Pack 38, or "mollies" has displayed the most overall persistence in their territory over time. This pack specializes in hunting bison.
\end{verbatim}

Problem 10. (3 points) Perform one analysis or exploration of your
choice on the \texttt{wolves} data. Your answer needs to include at
least two lines of code and not be a summary function.\\
I am going to find the largest August pack size recorded in each park.

\begin{Shaded}
\begin{Highlighting}[]
\NormalTok{wolves }\SpecialCharTok{\%\textgreater{}\%}
  \FunctionTok{group\_by}\NormalTok{(park) }\SpecialCharTok{\%\textgreater{}\%}
  \FunctionTok{summarise}\NormalTok{(}\AttributeTok{largest\_pack =} \FunctionTok{max}\NormalTok{(packsize\_aug, }\AttributeTok{na.rm =}\NormalTok{ T)) }\SpecialCharTok{\%\textgreater{}\%}
  \FunctionTok{arrange}\NormalTok{(}\FunctionTok{desc}\NormalTok{(largest\_pack))}
\end{Highlighting}
\end{Shaded}

\begin{verbatim}
## # A tibble: 5 x 2
##   park  largest_pack
##   <chr>        <dbl>
## 1 YNP           37  
## 2 DENA          33  
## 3 GNTP          26.4
## 4 YUCH          24  
## 5 VNP            7
\end{verbatim}

\#SAVE AS PDF AND SAVE AND SUBMIT TO GRADESCOPE

\end{document}
